\PassOptionsToPackage{unicode=true}{hyperref} % options for packages loaded elsewhere
\PassOptionsToPackage{hyphens}{url}
%
\documentclass[]{article}
\usepackage{lmodern}
\usepackage{amssymb,amsmath}
\usepackage{ifxetex,ifluatex}
\usepackage{fixltx2e} % provides \textsubscript
\ifnum 0\ifxetex 1\fi\ifluatex 1\fi=0 % if pdftex
  \usepackage[T1]{fontenc}
  \usepackage[utf8]{inputenc}
  \usepackage{textcomp} % provides euro and other symbols
\else % if luatex or xelatex
  \usepackage{unicode-math}
  \defaultfontfeatures{Ligatures=TeX,Scale=MatchLowercase}
\fi
% use upquote if available, for straight quotes in verbatim environments
\IfFileExists{upquote.sty}{\usepackage{upquote}}{}
% use microtype if available
\IfFileExists{microtype.sty}{%
\usepackage[]{microtype}
\UseMicrotypeSet[protrusion]{basicmath} % disable protrusion for tt fonts
}{}
\IfFileExists{parskip.sty}{%
\usepackage{parskip}
}{% else
\setlength{\parindent}{0pt}
\setlength{\parskip}{6pt plus 2pt minus 1pt}
}
\usepackage{hyperref}
\hypersetup{
            pdfborder={0 0 0},
            breaklinks=true}
\urlstyle{same}  % don't use monospace font for urls
\setlength{\emergencystretch}{3em}  % prevent overfull lines
\providecommand{\tightlist}{%
  \setlength{\itemsep}{0pt}\setlength{\parskip}{0pt}}
\setcounter{secnumdepth}{0}
% Redefines (sub)paragraphs to behave more like sections
\ifx\paragraph\undefined\else
\let\oldparagraph\paragraph
\renewcommand{\paragraph}[1]{\oldparagraph{#1}\mbox{}}
\fi
\ifx\subparagraph\undefined\else
\let\oldsubparagraph\subparagraph
\renewcommand{\subparagraph}[1]{\oldsubparagraph{#1}\mbox{}}
\fi

% set default figure placement to htbp
\makeatletter
\def\fps@figure{htbp}
\makeatother


\date{}

\begin{document}

\begin{description}
\tightlist
\item[]
(时间管理很重要。电脑投屏显示每轮活动剩下的时间(分钟),让团队即时知道还有多少时间。例如最后时间到会显示
"TIME'S UP")
\end{description}

\hypertarget{ux56e2ux961fux7b2cux4e00ux8f6eux56deux987eux5b9eux4f8b}{%
\subsubsection{团队第一轮回顾实例}\label{ux56e2ux961fux7b2cux4e00ux8f6eux56deux987eux5b9eux4f8b}}

\begin{itemize}
\tightlist
\item
  团队分析迭代缺陷数据,缺陷是源自需求/设计/编码,得出以下分布表:
\end{itemize}

\url{文件:微信截图_20220316093044.jpg}

\begin{itemize}
\tightlist
\item
  按DRE公式,计算出各个过程的缺陷排除率,并把参数输进水晶球模型。例如
  第一行Quality 那行:10 (共有10个缺陷是源自需求);第二行:10\%
  (需求评审缺陷排除率是 10\%)。(本来迭代缺陷参数都标 Infosys 识别。)
\end{itemize}

\href{文件:2DreEstimateScreenshot_2021-12-01_212049.1.jpg}{文件:2DreEstimateScreenshot
2021-12-01 212049.1.jpg}\\
\href{文件:微信截图_20231031153027.png}{450px}

\href{文件:4reworkByPhaseScreenshot_2021-12-01_214838.1.jpg}{450px}

\begin{itemize}
\tightlist
\item
  从团队工时表估算各过程的缺陷返工工作量。例如 最下面一行Quality
  那行输入:\$20,000 (因为验收测试缺陷平均修复工时是20,假定每工时成本为
  \$1,000);系统测试那行:\$5,800(因为系统测试缺陷平均修复工时是5.8)
\end{itemize}

\href{文件:微信截图_20231031153355.png}{450px}

\begin{itemize}
\tightlist
\item
  估算质量总成本:水晶球会依据质每过程的缺陷分布和单位成本分布,预估质量成本的分布。按本来迭代需求引入缺陷数=10(3),需求评审缺陷排除率等于10\%(4),设计引入缺陷数=20(2),设计评审排除率=10\%(4),代码引入缺陷书=34(4),代码评审排除率=18\%(1),单元测试缺陷排除率=35\%(2),系统测试缺陷排除率=88\%(1),验收测试缺陷排除率=99\%(3)来估算质量成本分布。结果如下:(括号里的数字代表在模型参数表里那一栋,从1至4。)
\end{itemize}

\href{文件:startQUA.PNG}{450px}

从上图看到质量成本的分布, 90\% percentile = \$365,043

\begin{itemize}
\tightlist
\item
  团队经过根因分析找出纠正措施包括:

  \begin{itemize}
  \tightlist
  \item
    完善需求评审检查单,估计可以把评审缺陷排除率从10\%提升到80\%,但因为要完善检查单评审工时也会从本来的23人时增加到40人时。
  \item
    使用审查方式评审后台设计,估计可以把设计评审缺陷排除率从10\%提升到69\%,但评审工作量也会从48人时增加到65人时。
  \item
    使用代码走查方式,估计可以把代码评审缺陷排除率从18\%提升到35\%,但评审工作量也会从12人时增加到20人时。
  \end{itemize}
\end{itemize}

利用水晶球模型比较以上各种不同的配搭,选出总成本最低是配搭组合。在水晶球参数输入表格,在需求评审、设计评审和代码评审都加上新方法的工时和质量,在模型选最优配置需求评审可选方法(3、4)设计评审方法可选(3、4)代码评审可选方式(1、2)让模型从2
x 2 x 2 = 8种可选搭配中比较,挑选哪个搭配的总成本最低。

下图显示一直水晶球优化的过程,和最终的最``佳''配搭:

\href{文件:YH.PNG}{450px}

从上图看到最佳搭配是:

\begin{itemize}
\tightlist
\item
  代码评审(2)采用走查方式
\item
  设计评审(3)采用审查方式
\item
  需求评审(4)不变
\end{itemize}

预测模型也会展示总成本的分布如下:

\href{文件:YH-OG.PNG}{400px}

\begin{itemize}
\tightlist
\item
  预测模型依据我们输入的缺陷参数,得出每个过程的缺陷范围,例如系统测试缺陷数95\%
  上下限范围是15.2 至 19.1。例如单元测试缺陷数95\% 上下限范围是9.1至11.9
\end{itemize}

\href{文件:YH-ST.PNG}{400px}

\href{文件:YH-UT.PNG}{400px}

\begin{itemize}
\tightlist
\item
  如果我们把优化后的缺陷分布(右图)与本来迭代的缺陷分布(左图)比较,能看出改进后,本来在系统测试才发现的缺陷大多数可以预先在评审暴露,降低返工工作量,也可以预估评审缺陷范围,如果实际评审缺陷数未能达到范围,便可预警,尽快纠正。
\end{itemize}

\href{文件:MTKL.png}{450px}

如果没有预测模型,我们只能主观估计会降低,有了预测模型,便可以估计缺陷分布的变化,也能帮我们挑选最佳搭配,并预估下一轮的缺陷分布范围。

\textbf{解读预测模型最佳搭配选择:}\\
模型以总成本最低可以防止局部最优。例如,需求评审没有选用排除率高的新方法,因为新方法会使用较大评审工作量。如果用质量最优(质量成本最低)便必然会读选用缺陷排除率最高的方法,但用总成本来比较便可以综合考虑工作量(Effort)
和 质量(Quality),避免局部最优。

今次有3种新方法比较,共8个配搭。假如3类评审都有两种新方法可选,我们便要比较
27 (3 x 3 x
3)配搭,每个配搭都单独模拟预估成本,然后统一比较,会非常费时,水晶球的最优功能能帮助快速选出最佳配搭。

质量经理:做了第一轮迭代后,以后的迭代有什么要注意?\\
我:团队累积了多轮迭代数据,便可以开始分析数据,并开始建立标杆。\\

收集数据很花精力,所以必须只收集与改进目标相关的数据,但如何能联想到那些度量使相关?GQM利用问题让我们找到相关的度量。

1.制定目标

\href{文件:Gqm1Screenshot_2023-11-01_113815.jpg}{500px}

2.针对目标提出问题

\href{文件:Gqm2Screenshot_2023-11-01_114014.jpg}{500px}

3.针对问题选择度量项

\href{文件:Gqm3Screenshot_2023-11-01_114048.jpg}{500px}

4.如何收集数据

\begin{itemize}
\tightlist
\item
  谁来收集
\item
  什么时候收集?
\item
  如何能有效收集数据并也保证正确?
\item
  谁是数据分析的对象?
\end{itemize}

5.收集、确认和分析数据,并反馈,让项目组制定纠正措施

6.改进后分析数据,判断达标的程度

7.做反馈,让所有利益相关者讨论

有了目标与度量项,我们便可以制定度量计划。除了目标与度量项外,计划也包括如何收集与分析数据。从下图度量计划模板看到里面包括以上四点(深绿色圆圈):

\href{文件:13_MA_plan_Screenshot_2023-10-26_211815.jpg}{650px}

\end{document}
