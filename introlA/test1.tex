\chapter{如何改善} % Introduction chapter suppressed from the table of contents

\hypertarget{ux67d0ux8f6fux4ef6ux5f00ux53d1ux516cux53f8ux654fux6377ux5f00ux53d1ux8fc7ux7a0bux6539ux8fdbux6848ux4f8b}{%
\subsection{某软件开发公司敏捷开发过程改进案例}\label{ux67d0ux8f6fux4ef6ux5f00ux53d1ux516cux53f8ux654fux6377ux5f00ux53d1ux8fc7ux7a0bux6539ux8fdbux6848ux4f8b}}

\hypertarget{ux6708}{%
\subsubsection{5月}\label{ux6708}}

张工是公司的中层管理,管理好几个开发团队,有五位项目经理向他汇报。\\
他听说老同学的团队都开始用敏捷开发,很感兴趣,并参加了一些敏捷的交流会,觉得可以解决很多传统瀑布性开发的不足,尤其是可以快速交付给客户。\\
他要求部门经理全面推动敏捷开发,对团队成员进行相关培训,例如,SCRUM
Master 内部培训。\\
开始时,张工上级部门经理有些怀疑,问:``后面那些工程文档都不做,会不会影响质量和交付?客户都是专门做电信的,不缺钱,但是对质量要求很高。''\\
张工便解释说,``只要利用敏捷把过程变成迭代,快速交付,改善工程的问题不难,主要是人的问题。''\\
部门经理听到敏捷可以比以前更快速交付,之前客户经常因为延误而不满,他希望可以改变这现状,就答应了。

\hypertarget{ux6708-1}{%
\subsubsection{8月}\label{ux6708-1}}

开发组长王工周五下班后与朋友喝酒,开开心心说:
``太兴奋了。研发部门经理决定全面推动敏捷开发SCRUM;我们团队刚参加了两天培训,真正对应我们以前的传统瀑布式开发的种种问题,我们会2周一次迭代,快速反馈,我们会定期小步向客户发版,我们会与用户经常交流,获得他们的反馈。\\
现在团队合作不像以前只按工种工作,也会跟产品经理、业务方面更充分合作,给客户带来更高价值。\\
工作方式也改变了,以前要写需求、规格说明书,现在简单化成用户故事和产品需求卡片,以前我们要做过详细项目计划甘特图,现在用改成用燃烧图和白板。每天用便利贴去写要开发什么东西贴在白板上面,开始的时候,贴的越多感觉越敏捷,我们改成叫
SCRUM
team,有一系列的海报围绕我们。我们也没项目经理了,自己管理自己。本来的部门经理现在变成产品负责人,敏捷开发方式让我们团队自己做决策,不仅仅是技术方面,项目相关的也由我们项目组一起讨论决定。\\
解除了以前`瀑布式开发'的种种烦恼,这一切太完美了。``\\

\hypertarget{ux6708-2}{%
\subsubsection{9月}\label{ux6708-2}}

''你们团队学完敏捷SCRUM后,项目如何?''\\
王工充满自豪地说:\\
``我们培训后就SCRUM的方法,定每两周一个冲刺,每次冲刺前都会用故事点来估算每个功能多大,然后按本次冲刺的资源,估计可以完成多少功能?\\
然后用白板来监控模块完成的情况,哪些在开发中,哪些已经完成,团队和管理者都可以一目了然,不用像以前天天问我们了。我们每天早上也按照SCRUM的规定站立会议,每人说自己完成了哪些任务,今天做什么。\\
大家都很兴奋,确实跟以前瀑布的做法不同。''

\hypertarget{ux6708-3}{%
\subsubsection{10月}\label{ux6708-3}}

''你们项目如何?''\\
王工听完,想了一下,然后说:\\
``我们本应上周要完成一次冲刺后的割接上线,但被推到下次了。``\\
''为什么?''\\
王工说:
''我们按培训学到的做冲刺计划会,按照产品的待办事项列表,团队利用扑克牌一起估算每一事项所需要的时间。我们总共八位开发人员,其中有一半是刚毕业不久,但大家刚上完培训,很有信心,虽然技术主管张工对我们出来的估算有些顾虑,觉得我们太理想,但大家刚培训完敏捷,张工也希望让部门经理尽快看到敏捷开发可以加快速度,我们就按这`进取式'估算开展2周冲刺。\\
但因新人多,编码水平有限,虽然大家已经尽快把开发出来的代码交给系统测试人员手工测试,依据测试发现的缺陷修正再测试,但越来越接近答应客户的2周割接上线时间,但是还是很多BUG没改好,最后几天,基本就天天加班,最终到验收时,仍然有不少问题,最终割接前测试,还是不能达到客户要求的水平,没办法,未能上线。\\
大家确实都尽力冲刺了,但未能达到我们本来希望的结果。``

\hypertarget{ux6708-4}{%
\subsubsection{11月}\label{ux6708-4}}

部门经理之前收到客户总监电话,投诉一些技术缺陷,导致好几次不能按计划上线,问为什么正在交付的软件质量变差了?\\
张工被问到是什么原因时无法回答,只能说立马回去探索原因,尽快汇报,但心里想:
``开始敏捷后,因为快速迭代,以前要做概要设计、详细设计的过程反而被忽略了,导致有些写出来的代码,后面就很很难适应快速的变化修改,导致要不然就功能做不出来。
因要赶时间,可以按客户的要求时间交付的话,由于本身代码不好,只可以临时凑凑,不长久。''\\
张工从部门经理办公室出来后,找其中一位项目经理李工喝茶,回顾一下发现项目团队对这次敏捷SCRUM的改革有意见。例如上层为了更快速交付,实现敏捷可以快速交付承诺,把一些本来不太可能的进度时间压到团队去,完全不是本来的那种自主团队管理的概念。出现问题多了,就请了敏捷教练过来辅导,但SCRUM的教练也缺乏软件工程的基础,只懂项目管理过程。所以他们也解决不了软件相关的问题。只是把精益管理怎么做迭代,怎么做回顾这些基本过程再解读一下,解决不了实际问题。\\
李工:\\
''因为我们做这块业务已经很多年了,本来业务的变化不多,只是一些小的功能改动,所以开发人员尽量不去动核心的代码,怕改动了反而会影响投产,切割不了。但有时为了满足一些新功能,继续在老代码的基础上去写,这种做法效率很低,也不长久,估计一两年后会难以运行了,我们会被迫重写整个产品,而老代码开发人员大部分都离开了,后面的代码维护变得非常困难,即使用敏捷也解决不了这个问题。''

无论张工或李工也没有能去总结出什么好的解决方案。现在推行敏捷才刚刚三个月,绝不能打退堂鼓,回到本来的状态。但应怎么解决敏捷带来的问题?挽回部门经理与客户的信心呢?\\

\begin{description}
\tightlist
\item[]
= = =
\end{description}

从上面的SCRUM案例看到,本来管理层希望利用敏捷开发,加快软件开发的交付,减少延误,令客户更满意。但因为只注重项目进度是否延误,但团队没注意如何改善软件开发本身的质量,也因为团队成员能力不足,开发出来的软件缺陷比以往还多,导致后面大量返工,恶性循环,后面更导致延误和客户投诉。\\
因为软件本身设计有问题,导致软件难以修改,开发人员都不敢改动任何代码,怕可能会引起系统崩溃。

怎样可以确保开发出来软件的质量?

敏捷开发有很多种方法(SCRUM
只是其一),因为目的不仅仅是管好项目进度,也要确保软件产品的质量。所以SCRUM
只包括项目管理部分,不全面,反过来,例如极限编程(XP eXtreme
Programming),因它的发明者Kent BECK
本身是一位精通面向对象的编程员,所以XP不仅仅关注项目管理,也包含编程的最佳实践。下图是
Ron Jeffery 把XP的重点画成从外到内3层:

%\href{文件:cleanagile_f1.8.jpg}{500px}。

\includegraphics[width=6cm]{cleanagile_f18.jpg}

SCRUM 只包含了外层的部分,缺乏中间和内层元素。
按XP的12实践(详见附件)都做到了便可以解决张工的问题吗?\\

\hypertarget{ux5982ux4f55ux6539ux5584}{%
\subsection{如何改善}\label{ux5982ux4f55ux6539ux5584}}

既然SCRUM方法有不足,XP方法能解决开发质量问题,是否团队学好XP便能帮助团队做好敏捷开发?\\
不一定,乔布斯离开苹果公司后,自己开创
NEXT,他接受访谈时总结了质量改进的重点:

\framebox{%
\begin{minipage}[t]{0.97\columnwidth}\raggedright
质量提升的道理其实很简单,是重复的过程:我们需要不断去看,有哪些无效的环节要省略,哪部分要重新设计,不断试验、提升,就这么简单。重点是所有的提升都应该是科学化的,有数据而不是泛泛而谈。

一般管理层的思路是:我是领袖,你们应该听我命令。但应该是反过来,让应如何做好的决定权利放在团队手上,做改进不需要请求管理层的批准。改进是工作的一部分,整个架构扁平化,自己管控日常过程,每位工程师应像以前的工匠,愿意花精力不断做好。然后能以自己最后做出的优质工艺、产物自豪。
\strut
\end{minipage}}

所以质量改进的原则:明确改进目标后不断完善的过程。这不仅仅适用于软件开发,也适用于其他,如工业生产用。

怎样才能不断完善,先问你以下关于汽车公司的问题:

%\href{文件:TvsGcompScreenshot_2023-06-07_121806.jpg}{650px}

\includegraphics[width=6cm]{TvsGcompScreenshot_2023-06-07_121806.jpg}

T公司在六十年代,在销售、生产成本都远远不如G公司,但它每年生产率都一直提升。

请猜猜T和G是那家公司?
(提示:两家都是世界级汽车公司,现在你还能买到它们生产的汽车。)

\begin{description}
\item[]
\begin{description}
\tightlist
\item[]
= = = = = = = = = = = =
\end{description}
\end{description}

G 是美国通用; T 是日本丰田。

为什么丰田能从一家战后小公司能提升为世界最大 (\#) 的汽车公司?

\begin{description}
\item[]
\begin{description}
\tightlist
\item[]
(\# :
2023年底,总收入可能是大众第一,丰田第二,但丰田销售数量世界第一。)
\end{description}
\end{description}

\hypertarget{ux4e30ux7530ux6545ux4e8b}{%
\subsection{丰田故事}\label{ux4e30ux7530ux6545ux4e8b}}

\href{文件:ToyotaPresidentPicture1.png}{200px}\\
丰田喜一郎先生 (创始人)\\
%\href{文件:大野耐一.png}{200px}

\includegraphics[width=6cm]{大野耐一.png}

二战后50年代,丰田汽车规模很小,经营很困难,在破产边缘。
但丰田创始人喜一郎先生明白美国生产线的弊端,意识到未来的汽车生产必须是Just-In-Time::

\begin{itemize}
\tightlist
\item
  每一辆都是按客人订单订制:例如,颜色,配置,左右钛等
\item
  从钢材原料开始,整个生产线,零等待,零浪费
\end{itemize}

\begin{description}
\item[]
\begin{description}
\tightlist
\item[]
每个工作步骤所需配件按生产需要到达
(不晚,也不早到),把生产过程中的配件降到零
\end{description}
\end{description}

\url{文件:远景1.png}

喜一郎先生安排总工大野耐一去美国汽车公司考察。

\framebox{%
\begin{minipage}[t]{0.97\columnwidth}\raggedright
为什么福特Model-T 出名?它是第一辆一般人买得起的汽车
首辆现代汽车是由德国奔驰(Benz)于1885制作,
但是这些第一代汽车都非常昂贵。 老福特是一名工程师,对机械特别感兴趣,
白天是爱迪生公司的总工程师,到了下班后就研究汽车发动机
经过十年的不断优化,最终开发出4冲程(Quadcycle, a
self-propelled)自动汽车发动机 他的发明大大降低了汽车组件的成本 1908
Model-T 首次推出市场,售价是850美金,虽然已经比同期的其他厂家汽车便宜,
但还是超出一般工人家庭可以负担的水平。

为了进一步降低成本,老福特觉得必须想其他办法。
汽车组装一直都是作坊式运作, 所以组装汽车需要超过12个工时。
为了降低组装时间,必须要学其他工厂,
如啤酒,麦粉厂,它们都已经是流水式生产线生产(Assembly Line)
福特把整个Model-T组装拆分成84步骤, 并培训工人只做一项工作,
还聘用了科学工程师TAYLOR先生 设计整个流程,提高效率,
最终可以一个半小时完成组装一辆汽车。
生产线生产帮助福特公司把成本和售价降下来, 在15年期间(1913 -
1927),福特公司总共生产了15,000,000辆 Model-T。
虽然大量生产能直接降低生产成本, 但这种做法也要付出代价:

\begin{itemize}
\tightlist
\item
  为了生产所有都是同一个款式同一个颜色,黑色
\item
  为了提高生产的效率,也增加了组件的库存量
  (为了提高效率,降低成本,大部分组件都要批量生产(Batch
  production),例如轮胎。例如上面右下图看到工厂里存在大量轮胎。)
\end{itemize}

%\href{文件:FordModelT_2023-06-10_100431.jpg}{600px}

\includegraphics[width=6cm]{FordModelT_2023-06-10_100431.jpg}

\textbf{Federick TAYLOR与科学管理(Scientific Management)}\\
%\href{文件:Weisbord_P34.jpg}{200px}

\includegraphics[width=6cm]{Weisbord_P34.jpg}

十九世纪末美国很多工业快速扩张,Taylor先生发现当时很多工厂生产力很低,例如钢铁厂。
他是很落地、实干的人,他就想有什么好方法可以让提高生产力,工人也多赚些钱。他发现很多工作都未设计好,未能好好利用工人,他针对工作本身做很多科学研究,测量每个工作应该多长时间,怎么做最好,然后按他的最佳设计把工作细分,也同时要求公司提高员工的奖励,希望工人看到因科学管理,提高生产,个人也受益。1890年,他加入了Bethlehem
Steel,钢铁业的大公司,帮它设计了很多奖励制度,也设定了一些叫工业工程师(Industrial
engineers)岗,发现生产力可以提升30-40\%。后来公司被收购,他也被辞退,他便成为顾问,当时都没有顾问这个职业,他可算是全球首位工程顾问。
\strut
\end{minipage}}

总工大野耐一先生,
从美国的超市(非汽车公司)得到如何做Just-In-Time的启发,回国后就开始在丰田致力推动。
开始时,很多人都觉得Just-In-Time 这个愿景好像是远不可及的梦想。

今天,现代汽车生产都基本做到了当年喜一郎先生和大野耐一先生的梦想。(详见附件`现代汽车生产')

\hypertarget{ux6c34ux9762ux4e0bux7684ux7ba1ux7406ux601dux8def}{%
\subsection{水面下的管理思路}\label{ux6c34ux9762ux4e0bux7684ux7ba1ux7406ux601dux8def}}

%\url{文件:冰山.png}

\includegraphics[width=6cm]{冰山.png}

为什么丰田能成功地把汽车生产做到 Just-In-Time
,超越西方的巨头,使日本汽车制造过程成为世界"标准"?
但我们不能单从表面看这"系统"的方法和技巧,例如大家都熟悉的看板管理(他的竞争对手通用、福特、肯定都学过),更重要是了解背后的管理思路。我们看看下面丰田九条原则,开始了解水面下的"系统":\\
\# 以忙碌为耻

\begin{enumerate}
\tightlist
\item
  培养人才 - 逼他们动脑筋
\item
  从心里相信``大家的力量''
\item
  不以``我们公司''作主语
\item
  标杆管理 (Benchmarking)
\item
  容易实现的目标不是好目标
\item
  根因分析
\item
  成长比成功更重要
\item
  所有人参与改进
\end{enumerate}

\hypertarget{ux4ee5ux5fd9ux788cux4e3aux803b}{%
\subsubsection{以忙碌为耻}\label{ux4ee5ux5fd9ux788cux4e3aux803b}}

\begin{itemize}
\tightlist
\item
  不吝惜智慧,但要吝惜汗水
\end{itemize}

\framebox{%
\begin{minipage}[t]{0.97\columnwidth}\raggedright
{把``动作''转变为 ``工作''}
以前,当大家批评某机关的工资太高时,职员会以上班时间很长,
``一直在努力工作''为由来反驳人们的批评。这其实是一种对于``有动作''和``在工作''的混淆。不管上班时间有多长,如果没能够创造出利润,那么就不能称之为工作,也不能称之为一直在努力。
丰田是把``动作''和``工作''分开考虑的。\\在丰田看来,
``就算一直在动也不代表那个人在工作''。
省掉徒劳动作,把``在动着''转化为``在工作''。\\大野耐一先生曾经问过年轻员工:
``每天工作一小时左右,你们能做到吗?"
听了这句话后,有人抱怨``算上加班时间,我们一天工作九个小时,他那句话是什么意思?"确实,他们在公司待了很长时间,但如果把``有动作''和``在工作''分开考虑,九个小时中,真正在工作的时间可能只有一小时。大野先生指的是这一点。\\ 关键的不在于流着汗在公司转了多长时间,而要把自己的工作区分成``徒劳作业''和``有价值作业''。
\strut
\end{minipage}}

因缺乏数据,很难判断这些软件开发人员每天有效生产多少?什么因素妨碍员工生产率提升?\\
\{\textbar{} class="wikitable" \textbar{}
二战后,日本经济萎缩,丰田辞退了不少员工;1950年,朝鲜战争,需要大量增产,但丰田选择了只增加设备,而不增加人手,大野先生也借这机会,完善丰田生产方式,成功找到了以不增加人手为前提的增产办法。

\textbar{}\}

\hypertarget{ux57f9ux517bux4ebaux624d---ux903cux4ed6ux4eecux52a8ux8111ux7b4b}{%
\subsubsection{培养人才 -
逼他们动脑筋}\label{ux57f9ux517bux4ebaux624d---ux903cux4ed6ux4eecux52a8ux8111ux7b4b}}

\framebox{%
\begin{minipage}[t]{0.97\columnwidth}\raggedright
{人了不起的智慧}
所谓丰田生产方式其实就是要建立起一种体制,把``人了不起的智慧''引导出来,使得这些智慧能在生产一线得到充分发挥。所以说``丰田生产方式源自人的智慧''。
\strut
\end{minipage}}

企业相信员工的智慧以后,员工们的干劲、使命感、责任感都会随之而生,最重要的是,员工们会因此逐渐对工作抱有自豪感。一旦员工们的意识改变了,理所当然地,企业的竞争力也将获得很大的提高。\\
丰田生产方式是TQM (Total Quality
Management全面质量管理)的最佳例子,TQM强调专注客户、持续改进、以数据说话、员工参与等,丰田生产方式覆盖了
TQM 原则的 七项 (除了战略)

%\href{文件:TQM_1.2.png}{500px}

\includegraphics[width=6cm]{TQM_12.png}

\hypertarget{ux4eceux5fc3ux91ccux76f8ux4fe1ux5927ux5bb6ux7684ux529bux91cf}{%
\subsubsection{从心里相信``大家的力量''}\label{ux4eceux5fc3ux91ccux76f8ux4fe1ux5927ux5bb6ux7684ux529bux91cf}}

\begin{itemize}
\tightlist
\item
  不是靠``一个不平凡的人'',而是依靠``一百个平凡人''来创造亮眼的成绩
\item
  生产产品就是培养人才
\end{itemize}

\begin{description}
\tightlist
\item[]
``做事业最关键是人.......`培养人才'为基础''总裁丰田英二先生
\end{description}

\begin{description}
\tightlist
\item[]
有一位丰田员工被问到``丰田生产方式到底是什么''的时候,这样回答:
``就是在人的智慧建起的基础上,立起了自动化和及时生产这两根支柱。''\\

\textbf{他所说的``人的智慧''是指``在一线工作人员的智慧''}。\\
\end{description}

\hypertarget{ux4e0dux4ee5ux6211ux4eecux516cux53f8ux4f5cux4e3bux8bed}{%
\subsubsection{不以``我们公司''作主语}\label{ux4e0dux4ee5ux6211ux4eecux516cux53f8ux4f5cux4e3bux8bed}}

\begin{itemize}
\tightlist
\item
  不是从``专业''的角度,而是从``顾客''的角度生产产品
\end{itemize}

\begin{description}
\tightlist
\item[]
创业大忌 -
闭门造车,所以丰田的原则,对客户有用就一定做出来,但对客户无用,或者不想要,就绝不生产。\\
\end{description}

%\url{文件:丰田p3.png}

\includegraphics[width=6cm]{丰田p3.png}

要创新必须从``顾客''的角度看,不单从工程师的视角看不仅适用于丰田和汽车制造:

\framebox{%
\begin{minipage}[t]{0.97\columnwidth}\raggedright
征服太空,踏上月球 六七十年代,美国开始阿波罗太空计划。
首批宇航员强烈要求工程师加上窗户,逃生门,可人手控制等
(现代我们会觉得这些是太空船 (spacecraft) 基本要求, 但当时是闻所未闻)

1980,一位NASA工程师回顾当时工程部的想法:
``你们又不是火箭专家,航天专家,只是飞行员。
希望可以在火箭驾驶舱人工控制火箭飞行,开玩笑!
太极端,你们可不了解成本多高,我们应可以很轻易用工程的理由拒绝。''
\strut
\end{minipage}}

\hypertarget{ux6807ux6746ux7ba1ux7406-benchmarking}{%
\subsubsection{标杆管理
(Benchmarking)}\label{ux6807ux6746ux7ba1ux7406-benchmarking}}

很多公司都会用百分比来设定改善目标,但改善了百分比,不一定代表质量有改善,例如:
某快递公司去年送包裹未按时送达占8\%,今年是6\%,好像已经改善了2\%。
但去年托运的数量是500万件,所以8\%是4万件;
今年的数量是750万件,所以今年的6\%就是4万5千件。

所以今年包裹延误增加了5000,这样非但没有任何改善,反而服务变得更差了。所以更重要是看数字本身。容易达到的目标,不是好目标。所以丰田贿一般选行业里最强的对手作为基准(标杆)。

1965年,美国通用汽车公司是世界顶尖。例如: 销售额的规模,丰田与通用是 1
: 60 完全不在同一个层次上

成本:丰田对通用是 1 : 0.5 成本是通用的一倍

丰田把当时顶尖的通用公司当成了自己进行标杆管理的对象

如果丰田某零件的成本价格是 1000日元,通用是400日元
丰田会把通用的400日元,作为基准成本价,把原料价定为400日元,把差额600作为``不必要花费'',让团队立马改善,尽早达到标杆。

丰田(如下图)很注重各种标杆,例如内部标杆、竞争性标杆等。软件开发也应该同样利用数据来制定量化目标。
%\url{文件:丰田p1.png}

\includegraphics[width=6cm]{丰田p1.png}

\hypertarget{ux5bb9ux6613ux5b9eux73b0ux7684ux76eeux6807ux4e0dux662fux597dux76eeux6807}{%
\subsubsection{容易实现的目标不是好目标}\label{ux5bb9ux6613ux5b9eux73b0ux7684ux76eeux6807ux4e0dux662fux597dux76eeux6807}}

\begin{itemize}
\tightlist
\item
  不是``削减一成'',而是通过``取消一个零''来发现浪费
\end{itemize}

把原先需要三小时的工作,改用三分钟完成。\\
听到上司这个要求,你该怎么回答呢?多数人会常识性地回答:``这太强人所难了''、``绝对做不到''。\\

\framebox{%
\begin{minipage}[t]{0.97\columnwidth}\raggedright
丰田 \\
单分换模(Single Minute Exchange of
Die)''例子:\\ 1965年,丰田汽车在推行丰田生产方式时遭遇一个瓶颈:装置更换时间太长,其中特别是500吨冲压机和1000吨冲压机的模具,更换时间长达\\4小时。如果不缩短这两个模具的更换时间,就不可能实现多品种少量生产方式。\\ 挑战由大野耐一先生统领,在生产管理的先行者,新乡重夫先生的指导下,分两个阶段展开。\\
 改善开始之前,新乡先生凭借自己多年的工作经验,了解到装置更换有两种方式。\\
 ①内部装置更换------必须在机器停下来以后,才能进行的装置更换。\\
 ②外部装置更换------能在机器运转过程中,或是在运转起来以后,进行的装置更换。\\
 要缩短时间,把内部装置更换和外部装置更换清楚地分开来是一个关键。能在外部装置替换作业中进行的工作,就全部在外部装置替换过程中实施。同时,分别对内部装置替换和外部装置替换进行改善。通过这种方法,装置更换时间缩短为一个半小时。\\
 完成这一改善花费了半年时间。通常能有这样的成果就可以告一段落了。但是,大野先生仍要求进一步缩短时间。\\
 他要求把更换过程``缩短为3分钟!''通常通过改善能把原先的2\textasciitilde{}4小时缩短为一个半小时,就可以很满意地说``已经很好了''。但是,大野先生不这么想。\\
 他认为:
``既然能缩短到这个水平,那么继续改善肯定可以把时间缩短为3分钟。''
\\改变汽车生产的单分换模的秘密\\对于这一要求,在以新乡先生为中心的技术小组中,自然有人提出``3分钟绝对干不完''。但是,新乡先生认为:
``如果能把内部装置更换全部转化为外部装置更换,
3分钟也不是不可能\\之后,他着手对多达100个以上的项目进行了改善。
\\ 首先,进一步对内部装置更换和外部装置更换进行细分,彻底把内部装置更换转化为外部装置更换。同时,想方设法对各种切割工具和模具进行设置,使得更换时用一个动作即可完成。此外,在紧固件上也动了很多脑筋。这样,终于创造出了无数项不花时间、能够简单完成同时可以在作业时保持稳定的改善。\\
紧接着,对作业顺序反复进行改善,实施标准化(制定没有多余工序的作业标准)。这样,在挑战进行了三个月后的某一天,真的只要3分钟就能完成了!这个结果让所有人都大吃一惊。
\strut
\end{minipage}}

\hypertarget{ux6839ux56e0ux5206ux6790}{%
\subsubsection{根因分析}\label{ux6839ux56e0ux5206ux6790}}

(五个为什么 是一种找根因的方法 , 详见附件B)

\begin{itemize}
\tightlist
\item
  不停留在``原因''上,而要找出``真因'',彻底改善
\item
  不追究责任,而是追究原因\\
\end{itemize}

丰田的大野耐一先生说:

\begin{description}
\tightlist
\item[]
\emph{做到一半是不行的,只有一个期限,``到完成为止''}
\end{description}

当工程师报了有问题就必须要求工程师查找原因、提供数据,但很多时候这个工作并不简单,但大野耐一先生决不放弃,必须找到根因。

\hypertarget{ux6210ux957fux6bd4ux6210ux529fux66f4ux91cdux8981}{%
\subsubsection{成长比成功更重要}\label{ux6210ux957fux6bd4ux6210ux529fux66f4ux91cdux8981}}

\begin{itemize}
\tightlist
\item
  要培养人才,``改变体制''比``改变人''更有效
\end{itemize}

有些工厂只依赖张贴标语、海报,希望可以减少工地的事故发生率,但丰田不注重喊口号,而是动手干实事,包括机械保安、设备保安等

%\href{文件:ft_223.1.png}{500px}

\includegraphics[width=6cm]{ft_2231.png}

所谓5S指的是:整理、整顿、清扫、清洁及修养。无法。。。产品的。。。。整顿''。实例:


\framebox{%
\begin{minipage}[t]{0.97\columnwidth}\raggedright
B先生。。。。知道的。(p229 - 230)

B先生在晨会上问。。。。工厂吗?''
某工人轻轻说``怎么可能。。。白搭''。。。随之出现了变化。
\strut
\end{minipage}}

大野耐一在他的《现场经营》中写过:

\framebox{%
\begin{minipage}[t]{0.97\columnwidth}\raggedright
``如果仅仅觉得用起来很方便,或。。。。。。。。。。 (p227 - 228)

\texttt{~附加值更高的工作。}\strut
\end{minipage}}

第一个例子,说明工作环境的重要性,主管通过逐步改善工厂的环境,团队的士气与生产率都提升了。

从第二个例子例子看到成长不是仅仅按既定方向做,必须思考动脑筋,并了解背后的目的才能真正取得效果。

这些丰田的成长故事也适用于软件开发团队,
例如,有些软件开发团队盲目地想推自动化测试,
却没有想好哪一类测试应自动化,哪一类更合适手工测试,
最后因效果与投入不匹配,失败告终。

\hypertarget{ux6240ux6709ux4ebaux53c2ux4e0eux6539ux8fdb}{%
\subsubsection{所有人参与改进}\label{ux6240ux6709ux4ebaux53c2ux4e0eux6539ux8fdb}}

\framebox{%
\begin{minipage}[t]{0.97\columnwidth}\raggedright
{带诚意去赢得协作}
B先生所在的A公司曾以丰田生产方式为基础进行了生产改革。\\
 要进行生产改革没有技术部门的配合是行不通的。但是,任凭
B先生怎么要求,技术部门依然毫不合作。B先生实在没有办法了,只能向总裁要求:
``请加大我手上的权力,让我可以支配技术部。''总裁回答:
``你去给我请教了大野(耐一)先生以后再说!''\\
 于是B先生去找大野先生,在听他诉说了自己面临的窘境以后,大野先生对他说:
``你这一两天跟我一起去工厂转转吧!''并于百忙之中抽出时间带B先生参观了丰田的工厂以及附近的协作企业的工厂。这期间,大野先生什么话都没说,只在第二天下午问B先生:``怎么样,你明白了吗?''\\
 B先生回答:
``我觉得在工厂听到的关于厂长的改善事例,跟丰田方式所强调的重点好像不太一致。''听了B先生的回答以后,大野先生点头:
``就连我,也是一直都在忍耐的啊。工作并不是有权力就能解决问题的。要想得到对方的理解和信任,拿出诚意去找人家吧。''\\
 从那以后,
B先生再也不找``因为我手里没有权力''之类的借口,而总是带着诚意去找对方协商,不久以后,他成功地对A公司实施了生产改革。\\
 每当听到有人感慨``下属不听话''的时候,一位曾在丰田工作的人就会说:
``你要求自己的孩子"每天学习三小时'时,他会听话去学习吗?''\\
 对方的回答是:
``估计没用。''\\
 ``连自己的孩子都这样,更何况那些成年的员工呢?''
\strut
\end{minipage}}

\hypertarget{ux6539ux5584ux8d28ux91cf}{%
\subsection{改善质量}\label{ux6539ux5584ux8d28ux91cf}}

质量管理,与财务管理类似,同样也有质量策划、质量控制和质量改进,基于这大框架再细分:如何制定质量目标,如何来制定度量等。\\
质量策划包括:

\begin{enumerate}
\tightlist
\item
  设定目标,包括外部和内部目标
\item
  识别内部需求
\item
  依据客户需求制定产品的功能特征
\item
  制定产品和过程的目标
\item
  设计过程来达到这些目标,最后验证过程能力\\
\end{enumerate}

\textbf{质量控制和质量改进}\\
下图的左面是在过程之前的策划部分,例如发现缺陷比率为20\%,这就是过程的能力,这是策划的时候已经定好的,过程控制没有什么可以做,只是当缺陷有变化,比如特别高的时候,需要做一些措施返回本来的水平。例如希望把缺陷从20\%降到3\%,这就必须驱动一系列的改进计划。改进计划也必须按项目管理方法推行与监控,没有其它办法。\\
%\href{文件:JuranImprovementScreenshot_2022-10-23_211444.jpg}{500px}

\includegraphics[width=6cm]{JuranImprovementScreenshot_2022-10-23_211444.jpg}\\

所以如果希望利用敏捷开发,不仅仅是走迭代,确保进度没偏差,还要确保软件产品质量,也应该用裘兰博士的质量管理思路去看敏捷过程,才能更全面了解如何才可以确保软件开发的质量,也控制好交付工期,不延误,让客户满意。\\
最佳实践,如果没有定质量目标,配上度量衡量和策划,都只是空想,对长远提升公司文化、团队成员的习惯没有任何帮助。举例:参照上图,例如我们想改善团队的策划和估算。首先要识别客户,哪些是主要干系人
-
甲方有什么要求;内部部门经理,有什么要求。然后从他们的诉求变成过程的功能和特征,但如果特征只是描述,没有数字也没有意义,所以要配合可衡量的度量单位,和用什么方式去收集那些数字,然后依据目标定过程要怎么去做,怎样改。

不是单纯`空降'某敏捷流程(例如 SCRUM),便能加快团队的发布速度。
所以虽然极限编程里每一条实践都是最佳实践,也必须配合质量策划,和监控改进才会有效果。

\hypertarget{ux7ed3ux675fux8bed}{%
\subsection{结束语}\label{ux7ed3ux675fux8bed}}

虽然XP比SCRUM更全面列出敏捷团队的最佳实践,但还必须依赖团队持续改善才会有效果。
从丰田故事看到``系统''如何帮公司培养知识工作者,发挥人的无限智慧,
为公司增值。\\
汽车制造大部分利用自动化机器,但当今软件开发生产和质量,
非常依赖开发人员的水平, 所以我们更需要建立``系统'',帮助员工快速成长。

前面分享的九个丰田管理思路:

\begin{enumerate}
\tightlist
\item
  以忙碌为耻
\item
  培养人才 - 逼他们动脑筋
\item
  从心里相信``大家的力量''
\item
  不以``我们公司''作主语
\item
  标杆管理 (Benchmarking)
\item
  容易实现的目标不是好目标
\item
  根因分析
\item
  成长比成功更重要
\item
  所有人参与改进
\end{enumerate}

丰田汽车,从50年代开始,沿用裘兰博士 (Dr
JURAN)的质量管理思路,成为世界最大的汽车企业。

针对软件开发,如何基于以上质量管理、精益管理思路,配合敏捷开发最佳实践,提升产品质量与团队竞争力?

所以团队成员,包括团队组长与组员,的能力是过程改进的基础,不仅仅依赖领导人。

按质量大师裘兰博士定义,质量包括两部分:

\begin{itemize}
\tightlist
\item
  满足客户需要 (客户包括外部客户和内部客户)
\item
  没有缺陷\\
\end{itemize}

这定义不仅仅适合于制造业、也适用于服务业,IT业。\\
因软件开发团队都有缺陷管理系统记录缺陷数,我们主要讨论如何降低后期暴露的缺陷,让客户更满意的过程改进方法。例如在第五部分会详细说明如何做好量化迭代回顾,降低验收测试和系统测试的缺陷密度。

下一部分,我们探索`团队与自我改善基本功' 与成功要素。

\hypertarget{ux9644ux4ef6}{%
\section{附件}\label{ux9644ux4ef6}}

\hypertarget{ux73b0ux4ee3ux6c7dux8f66ux751fux4ea7}{%
\subsection{现代汽车生产}\label{ux73b0ux4ee3ux6c7dux8f66ux751fux4ea7}}

今天Just-In-Time已成为汽车制造的主流,例如:日产在英国牛津(Oxfordshire)专门生产mini
车的 工厂便能做到:

\begin{itemize}
\tightlist
\item
  从钢材原料到生产出汽车只需要24小时
\item
  整个生产线没有任何中间等候,每68秒出一辆
\item
  每天生产1000辆车
\end{itemize}

%\href{文件:NissanRobots-OIP.h8UAC7FBjlaCmA_mBepJjQHaEi.jpg}{500px}\\
\includegraphics[width=6cm]{NissanRobots-OIPh8UAC7FBjlaCmA_mBepJjQHaEi.jpg}\\
\textbf{挑战}:每一辆汽车都不同 -- 颜色,设备,左右驾驶座等

\begin{itemize}
\tightlist
\item
  生产线上每一辆汽车都按照客户需求订制
\item
  组件不早不晚按需求准时到达生产线
\item
  这便需要信息化系统把客户订单转换成生产信息
\end{itemize}

%\href{文件:NissanJIT_OIP.RQGKy67DWGTu-DQiOCqW2gHaEK.jpg}{500px}
\includegraphics[width=6cm]{NissanJIT_OIPRQGKy67DWGTu-DQiOCqW2gHaEK.jpg}\\

例如,生产线上每一辆的颜色都可以不同:\\
%\href{文件:NissanProdLineOIP.wJFmfMl7q2_V8JaQi4kQ8QHaE6.jpg}{400px}\\
\includegraphics[width=6cm]{NissanProdLineOIPwJFmfMl7q2_V8JaQi4kQ8QHaE6.jpg}\\

\hypertarget{xp}{%
\subsection{XP}\label{xp}}

\hypertarget{ux7f16ux7801ux5b9eux8df5-coding-practices}{%
\subsubsection{编码实践 Coding
Practices}\label{ux7f16ux7801ux5b9eux8df5-coding-practices}}

\hypertarget{cp1ux7b80ux5355ux5730ux7f16ux7801ux548cux8bbeux8ba1-code-and-design-simply}{%
\paragraph{CP1:简单地编码和设计 Code and Design
Simply}\label{cp1ux7b80ux5355ux5730ux7f16ux7801ux548cux8bbeux8ba1-code-and-design-simply}}

\begin{itemize}
\tightlist
\item
  To produce software that is easy to change 使软件易于更改
\end{itemize}

\hypertarget{cp2ux65e0ux60c5ux5730ux91cdux6784-refactor-mercilessly}{%
\paragraph{CP2:无情地重构 Refactor
Mercilessly}\label{cp2ux65e0ux60c5ux5730ux91cdux6784-refactor-mercilessly}}

\begin{itemize}
\tightlist
\item
  To find the code's optimal design 找到代码的最佳设计
\end{itemize}

\hypertarget{cp3ux5236ux5b9aux7f16ux7801ux6807ux51c6-develop-coding-standards}{%
\paragraph{CP3:制定编码标准 Develop Coding
Standards}\label{cp3ux5236ux5b9aux7f16ux7801ux6807ux51c6-develop-coding-standards}}

\begin{itemize}
\tightlist
\item
  To communicate ideas clearly through code 通过代码清晰地传达想法
\end{itemize}

\hypertarget{cp4ux5171ux540cux7684ux8bcdux6c47-develop-a-common-vocabulary}{%
\paragraph{CP4:共同的词汇 Develop a Common
Vocabulary}\label{cp4ux5171ux540cux7684ux8bcdux6c47-develop-a-common-vocabulary}}

\begin{itemize}
\tightlist
\item
  To communicate ideas about code clearly 清楚传达软件设计的想法
\end{itemize}

\hypertarget{ux5f00ux53d1ux5b9eux8df5-develop-practices}{%
\subsubsection{开发实践 Develop
Practices}\label{ux5f00ux53d1ux5b9eux8df5-develop-practices}}

\hypertarget{dp1ux6d4bux8bd5ux9a71ux52a8ux5f00ux53d1tdd-test-driven-development}{%
\paragraph{DP1:测试驱动开发TDD
Test-Driven-Development}\label{dp1ux6d4bux8bd5ux9a71ux52a8ux5f00ux53d1tdd-test-driven-development}}

\begin{itemize}
\tightlist
\item
  To prove that code works as it should 来证明软件正常工作:\\
\end{itemize}

\begin{description}
\tightlist
\item[]
- Test-first programming(prim practice\#)
\end{description}

\hypertarget{dp2ux7ed3ux5bf9ux7f16ux7a0b-pair-programming}{%
\paragraph{DP2:结对编程 Pair
Programming}\label{dp2ux7ed3ux5bf9ux7f16ux7a0b-pair-programming}}

\begin{itemize}
\tightlist
\item
  To spread knowledge, experience and ideas 传播知识、经验和想法:\\
\end{itemize}

\begin{description}
\tightlist
\item[]
- Pair Programming(prim practice\#)
\end{description}

\hypertarget{dp3ux96c6ux4f53ux8d1fux8d23ux5199ux597dux4ee3ux7801vs-ux53eaux987eux8651ux81eaux5df1ux7684ux4ee3ux7801-collective-code-ownership-vs-individual-own-code}{%
\paragraph{DP3:集体负责写好代码(vs 只顾虑自己的代码) Collective Code
Ownership (vs individual own
code)}\label{dp3ux96c6ux4f53ux8d1fux8d23ux5199ux597dux4ee3ux7801vs-ux53eaux987eux8651ux81eaux5df1ux7684ux4ee3ux7801-collective-code-ownership-vs-individual-own-code}}

\begin{itemize}
\tightlist
\item
  To spread the responsibility for the code to the whole team
  将写好代码的责任扩展到整个团队:\\
\end{itemize}

\begin{description}
\tightlist
\item[]
- Whole team(prim practice\#)

- Share code(corollary practice\#)
\end{description}

\hypertarget{dp4ux6301ux7eedux96c6ux6210-integrate-continually}{%
\paragraph{DP4:持续集成 Integrate
Continually}\label{dp4ux6301ux7eedux96c6ux6210-integrate-continually}}

\begin{itemize}
\tightlist
\item
  To reduce the impact of adding new features 降低添加新功能的影响:
\end{itemize}

\begin{description}
\tightlist
\item[]
- Incremental Design(prim practice\#)

- Single code base(corollary practice\#)

- Ten-minute Build(prim practice\#)

- Continuous Integration(prim practice\#)
\end{description}

\hypertarget{ux5546ux52a1ux5b9eux8df5-business-practices}{%
\subsubsection{商务实践 Business
Practices}\label{ux5546ux52a1ux5b9eux8df5-business-practices}}

\hypertarget{bp1ux5c06ux5ba2ux6237ux6dfbux52a0ux8fdbux56e2ux961f-add-a-customer-to-the-team}{%
\paragraph{BP1:将客户添加进团队 Add a Customer to the
Team}\label{bp1ux5c06ux5ba2ux6237ux6dfbux52a0ux8fdbux56e2ux961f-add-a-customer-to-the-team}}

\begin{itemize}
\tightlist
\item
  To address business concerns accurately and directly
  准确、直接地解决业务问题:
\end{itemize}

\begin{description}
\tightlist
\item[]
- Real Customer involvement(corollary practice\#)
\end{description}

\hypertarget{bp2ux8ba1ux5212ux6e38ux620f-play-the-planning-game}{%
\paragraph{BP2:计划游戏 Play the Planning
Game}\label{bp2ux8ba1ux5212ux6e38ux620f-play-the-planning-game}}

\begin{itemize}
\tightlist
\item
  To schedule the most important work 安排最重要的工作:\\
\end{itemize}

\begin{description}
\tightlist
\item[]
- Weekly cycle ; Quarterly cycle ; Slack (prim practice\#)
\end{description}

\hypertarget{bp3ux5b9aux671fux53d1ux5e03-release-regularly}{%
\paragraph{BP3:定期发布 Release
Regularly}\label{bp3ux5b9aux671fux53d1ux5e03-release-regularly}}

\begin{itemize}
\tightlist
\item
  To return the customer's investment often
  尽早交付,让客户看到投资回报:
\end{itemize}

\begin{description}
\tightlist
\item[]
- Incremental Deployment(corollary practice\#)

- Daily Deployment(corollary practice\#)
\end{description}

\hypertarget{bp4ux4ee5ux53efux6301ux4e45ux7684ux901fux5ea6ux5de5ux4f5c-work-at-a-sustainable-pace}{%
\paragraph{BP4:以可持久的速度工作 Work at a Sustainable
Pace}\label{bp4ux4ee5ux53efux6301ux4e45ux7684ux901fux5ea6ux5de5ux4f5c-work-at-a-sustainable-pace}}

\begin{itemize}
\tightlist
\item
  To go home tired, but not exhausted 回家时虽然很累,但不筋疲力尽:\\
\end{itemize}

\begin{description}
\tightlist
\item[]
- Slack (prim practice\#)
\end{description}

\hypertarget{sux6cd5}{%
\subsection{5S法}\label{sux6cd5}}

本来5S是用于工业生产,例如日本的生产工厂很注重洁净,东西要放在固定位置,容易找到。其实这对生产线员工起很重要的心理作用,想象如果整个环境都很脏,必然会影响人工做好的动力,如果东西乱放,也容易找不到。5S也不仅仅是用于生产,比如医院、酒店也采用这方式管理,比如每件要用的东西必须在固定位置放好,也可以用于个人管理,比如我以前常常在出差去客户时,经常忘记把一些东西,如鼠标,或插头。但后面我固定了每件东西都应放哪里,我走的时候收拾就确保那些东西不会遗漏掉,平常工作有一个洁净的环境,也能降低工作压力,提高工作效率。

\href{文件:5S_五常法_Screenshot_2023-08-03_211606.jpg}{400px}

\includegraphics[width=6cm]{5S_五常法_Screenshot_2023-08-03_211606.jpg}\\

\hypertarget{ux53c2ux8003-references}{%
\section{参考 References}\label{ux53c2ux8003-references}}

\begin{enumerate}
\tightlist
\item
  若松义人, '为什么是丰田:成为第一的方法和7个习惯'
\end{enumerate}


