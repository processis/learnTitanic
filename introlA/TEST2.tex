\PassOptionsToPackage{unicode=true}{hyperref} % options for packages loaded elsewhere
\PassOptionsToPackage{hyphens}{url}
%
\documentclass{book}        % from not too short p76 pagestyle
%\usepackage{fancyhdr}
%\pagestyle{fancy}
%ensure chapter section headngs in lower case
%
\usepackage{graphicx}
\DeclareGraphicsExtensions{.png,.jpg}
\usepackage{xeCJK}
\setCJKmainfont{SimSun}
\usepackage{lmodern}
\usepackage{amssymb,amsmath}
\usepackage{ifxetex,ifluatex}
\usepackage{fixltx2e} % provides \textsubscript
\ifnum 0\ifxetex 1\fi\ifluatex 1\fi=0 % if pdftex
  \usepackage[T1]{fontenc}
  \usepackage[utf8]{inputenc}
  \usepackage{textcomp} % provides euro and other symbols
\else % if luatex or xelatex
  \usepackage{unicode-math}
  \defaultfontfeatures{Ligatures=TeX,Scale=MatchLowercase}
\fi
% use upquote if available, for straight quotes in verbatim environments
\IfFileExists{upquote.sty}{\usepackage{upquote}}{}
% use microtype if available
\IfFileExists{microtype.sty}{%
\usepackage[]{microtype}
\UseMicrotypeSet[protrusion]{basicmath} % disable protrusion for tt fonts
}{}
\IfFileExists{parskip.sty}{%
\usepackage{parskip}
}{% else
\setlength{\parindent}{0pt}
\setlength{\parskip}{6pt plus 2pt minus 1pt}
}
\usepackage{hyperref}
\hypersetup{
            pdfborder={0 0 0},
            breaklinks=true}
\urlstyle{same}  % don't use monospace font for urls
\usepackage{longtable,booktabs}
% Fix footnotes in tables (requires footnote package)
\usepackage{multirow} %for tabular combine multirow
\IfFileExists{footnote.sty}{\usepackage{footnote}\makesavenoteenv{longtable}}{}
\setlength{\emergencystretch}{3em}  % prevent overfull lines
\providecommand{\tightlist}{%
  \setlength{\itemsep}{0pt}\setlength{\parskip}{0pt}}
\setcounter{secnumdepth}{0}
% Redefines (sub)paragraphs to behave more like sections
\ifx\paragraph\undefined\else
\let\oldparagraph\paragraph
\renewcommand{\paragraph}[1]{\oldparagraph{#1}\mbox{}}
\fi
\ifx\subparagraph\undefined\else
\let\oldsubparagraph\subparagraph
\renewcommand{\subparagraph}[1]{\oldsubparagraph{#1}\mbox{}}
\fi

% set default figure placement to htbp
\makeatletter
\def\fps@figure{htbp}
\makeatother
\date{}
\begin{document}               % plus the \end{document} command at the end.
\begin{titlepage}\thispagestyle{empty} \vspace*{3em}{\centering\Huge 软件开发过程改进 \par}\clearpage
\newpage \thispagestyle{empty} \mbox{} \cleardoublepage
\thispagestyle{empty} \vspace*{7em}{\centering\Huge 软件开发过程改进 \par}{\centering -- from individual to team to organization \par}\cleardoublepage

\thispagestyle{empty} \vspace*{\fill} \parbox{.8\textwidth}{\raggedright \scriptsize
\textit{impossible} publisher 2022

printed blindfolded

design: \LaTeX
}
\end{titlepage}
\clearpage \thispagestyle{empty}\cleardoublepage
\newpage % Make sure the following content is on a new page

%----------------------------------------------------------------------------------------
%	TABLE OF CONTENTS
%----------------------------------------------------------------------------------------

\tableofcontents % Prints the table of contents

%----------------------------------------------------------------------------------------
%	INTRODUCTION SECTION
%----------------------------------------------------------------------------------------



\begin{tabular}{|l|l|p{5cm}|l}
\mbox{驾轻就熟} & jià qīng jiù shú & 驾轻车,走熟路。比喻对某事有经验,很熟悉,做起来容易。 & 驾轻车,走熟路。比喻对某事有经验,很熟悉,做起来容易。 \\
\end{tabular}


%\usepackage{fontspec}
%\usepackage{array}
%\setmainfont{NSimSun}
%\setmainfont{SimSong}
%\XeTeXlinebreaklocale=ZH
%\XeTeXlinebreakskip=0pt plus 1pt

%\begin{tabular}{|l|l|p{5cm}|}
%驾轻就熟 & 
%jià qīng jiù shú & \raggedright\arraybackslash
%驾轻车,走熟路。比喻对某事有经验,很熟悉,做起来容易。\\
%\hline
%驾轻就熟 & jià qīng jiù shú &
%驾轻车,走熟路。比喻对某事有经验,很熟悉,做起来容易。
%\end{tabular}

%try CSDN LaTeX (table) code
\begin{tabular}{|c|c|c|c|c|}
\hline
1.0&2.0&3.0&4.0&5.0\\
\hline
6.0&\multicolumn{3}{c|}{敏捷开发颜黑}&7.0\\
\hline
8.0&9.0&10.0&11.0&12.0\\
\hline
\end{tabular}

\begin{tabular}{|c|c|c|}   \hline
          &\multicolumn{2}{c|}颜黑 \\ \cline{2-3}
\raisebox{1.5ex}[0cm][0cm] T1
          &BLACK    &BROWN                 \\\hline
EYE 颜黑    &68  &119X黑                \\\hline
\end{tabular}



% \usepackage{multirow} %inserted beforebegin document
\begin{tabular}{|c|c|c|}
\hline
1.0&2.0&3.0\\
\hline
6.0&\multirow{3}{*}{敏捷颜黑}&7.0\\
\cline{1-1}
\cline{3-3}
8.0&&9.0\\
\cline{1-1}
\cline{3-3}
10.0&&11.0\\
\hline
12.0&13.0&14.0\\
\hline
\end{tabular}

%insert 3rows 2 columns in the middle
\begin{tabular}{|c|c|c|c|}
\hline
1.0&2.0&3.0&4.0\\
\hline
6.0&\multicolumn{2}{c|}{\multirow{3}{*}{敏捷颜黑}}&7.0\\
\cline{1-1}
\cline{4-4}
8.0&\multicolumn{2}{c|}{}&9.0\\
\cline{1-1}
\cline{4-4}
10.0&\multicolumn{2}{c|}{}&11.0\\
\hline
12.0&13.0&14.0&15.0\\
\hline
\end{tabular}

2

\begin{tabular}{|c|c|c|}
\hline
心态&表现&对策\\
\hline
满足&满足现状&不打扰\\
\hline
否认&你说我担忧!胡说!&提问,支持,不能提建议\\
\hline
迷惑&乱七八糟,救命!&放眼未来\\
\hline
更新&有很多改进想法/建议&支持执行\\
\hline
\end{tabular}

4


%try CSDN LaTeX (table) code
\begin{tabular}{|c|c|c|c|}
\hline
Process 过程&\multicolumn{3}{c|}{天数Durations}\\
\hline
1&27&30&75\\
\hline
2&45&50&125\\
\hline
3&72&80&200\\
\hline
4&45&50&125\\
\hline
5&81&90&225\\
\hline
\end{tabular}

%try CSDN LaTeX (table) code
\begin{tabular}{|c|c|c|c|c|c|}
\hline
Process 过程&\multicolumn{4}{c|}{天数Durations}\\
\hline
1&27&30&75&37&80\\
\hline
2&45&50&125&61.66&13.33\\
\hline
3&72&80&200&98.66&21.33\\
\hline
4&45&50&125&61.66&13.33\\
\hline
5&81&90&225&111&24\\
\hline
6&23&25&63&31&6.66\\
\hline
7&32&35&88&43.33&9.33\\
\hline
8&41&45&113&55.66&12\\
\hline
9&63&70&175&86.33&18.66\\
\hline
10&23&25&63&31&6.66\\
\hline
\:&\:&500&\:&617.33&\: \\
\hline
\end{tabular}

6

\begin{tabular}{|c|c|c|}
\hline
\:&Assembly&PL/S\\
\hline
代码行数&17,500&5,000\\
\hline
工作量(月)&30&12.5\\
\hline
工作量(工时)&3,960&1,650\\
\hline
每月代码行数&583.33&400.00 \\
\hline
等同的Assembly代码行数&17,500&17,500\\
\hline
等同的Assembly人月&583.33&1,400.00\\
\hline
功能点&100.00&100.00\\
\hline
功能点/人月&3.33&8.00\\
\hline
\end{tabular}

\begin{tabular}{|c|c|c|}
\hline
\:&活动&成本(\%)\\
\hline
1&发现与修正缺陷&30\\
\hline
2&编码&25\\
\hline
3&支持类文档&20\\
\hline
4&会议沟通&15\\
\hline
5&项目管理&10\\
\hline
\:&总计&100\\
\hline
\end{tabular}


\begin{tabular}{|c|c|}
\hline
活动名称&描述\\
\hline
预约申请&客户发出预约请求到预约系统,预约信息包括:日期、时间、上车的位置和目的地位置。\\
\hline
接收预约&预约系统收集客户预约请求,并把预约数据记录在数据库。\\
\hline
检查是否有可用的车/司机&预约系统从预约数据库中查看是否有合适的车/司机,如果能找到合适的时间、日期,能配合预约请求的话,看看司机是否有空档。如能找到,在系统更新内部状态为是,否则为否。 \\
\hline
寻找其他可选&如果状态是没有,预约系统继续在数据库搜索有没有接近的预期时间,否则写没有合适请求的出发地点和目的地。 \\
\hline
提供预约信息&预约系统自动发出通知到客户,是否有合适空档或者提供接近的日期时间。 \\
\hline
处理预约&客户回复预约系统接受或拒绝,预约系统把反馈记录在预约数据库。如果反馈是接受,预约系统会继续统计客户详细预约信息。如果客户拒绝,预约系统就会回复收到,并终止过程。 \\
\hline
分派司机&如果客户接受,预约系统会指派司机到已确认的日期、时间、上车地点和目的地。预约系统把这个记录在数据库中,并发信息通知相关司机。 \\
\hline
接乘客&司机按约好的日期时间、上车的地点接乘客,然后发信息到预约系统,通知乘客已经上车。\\
\hline
完成预约请求&预约系统在数据库中记录已经完成整个过程。\\
\hline
\end{tabular}

7

\begin{tabular}{|c|c|c|}
\hline
估算步骤&专家估算&模型\\
\hline
评判不同信息的比重&主观判断&基于对历史数据的统计分析\\
\hline
从信息估算工作量&主观判断&基于正式重复性的过程和方程式 \\
\hline
回顾/讨论工作量估算/主观判断+分析过程,参考检查单、指南和专家意见&主观判断+分析过程,参考检查单、指南和专家意见,可能再主观判断更新工作量估算 \\
\hline
\end{tabular}


\begin{tabular}{|c|c|c|c|c|c|}
\hline
学生ID&经验水平(自估)&I&II&III&IV\\
\hline
1&0&14.3&3.7&2.25&2.25\\
\hline
2&Low&10&4&2.5&2.5 \\
\hline
3&Low&20&4.5&3.75&3.75 \\
\hline
4&Medium&3&5&4.5&3\\
\hline
5&Medium&2.5&4&3.2&3.2\\
\hline
6&Low&8&4.1&4.1&3\\
\hline
7&0&20&7&5&3.5\\
\hline
8&Medium&20&5&4&2.75\\
\hline
9&Medium&4.5&5&4.5&3.5\\
\hline
10&Medium&3&3&3&2.5\\
\hline
11&High&4.6&3.8&2.8&2.4\\
\hline
12&Medium&2&2.5&2.7&2.3\\
\hline
13&Low&6.5&3.5&2.5&2.5\\
\hline
14&0&18&6&5&3\\
\hline
15&Low&20&3&3&3\\
\hline
16&High&5&5.5&2.3&2.5\\
\hline
17&High&15&3.8&2.7&2.4\\
\hline
Mean 均值&-&10.4&4.3&3.4&2.8\\
\hline
Median&-&8&4&3&2.8\\
\hline
SD 标准差&-&7&1.1&0.9&0.4\\
\hline
\end{tabular}

\begin{tabular}{|c|c|c|c|}
\hline
团队 ID&II(团队)&III(团队)&IV(团队)\\
\hline
1&3.8&2.7&2.4\\
\hline
2&3.7&3.4&3.16\\
\hline
3&4&4.3&2.85\\
\hline
Mean 均值&3.8&3.5&2.8\\
\hline
Median&3.8&3.4&2.9\\
\hline
SD 标准差&0.1&0.7&0.3\\
\hline
\end{tabular}



\begin{tabular}{|c|c|c|c|c|c|c|}
\hline
Team ID&\# Pieces&对象人群岁数&团队人数&实际总工作量(分钟)&模型估计总工作量 (分钟)&偏差 MRE \\
\hline
1&346&12+&1&37&46.21&0.25\\
\hline
2&321&12+&1&48&45.39&0.05\\
\hline
3&261&7 to 12&1&32&43.19&0.35\\
\hline
4&345&12+&1&45&46.18&0.03\\
\hline
5&261&7 to 12&1&59&43.19&0.27\\
\hline
6&345&12+&2&114&73.48&0.36\\
\hline
7&346&12+&3&84&96.48&0.15\\
\hline
8&321&12+&4&104&114.90&0.10\\
\hline
\end{tabular}


12

\begin{tabular}{|c|c|c|}
\hline
项目组&原因&投票数\\
\hline
**雄&建议使用FindBugs代码扫描工具&1\\
\hline
张**&缺少系统业务培训&基1\\
\hline
高**&没有合适的代码扫描工具&2 \\
\hline
李*&希望公司制定缺陷数据分析标准&1\\
\hline
韩**&公司应提供合适的代码扫描工具&3\\
\hline
高*&公司应加强相关业务培训&2\\
\hline
*一铭&没有合适的代码扫描工具&4\\
\hline
孙**&建议使用有力的代码扫描工具&5\\
\hline
\end{tabular}


\begin{tabular}{|c|c|}
\hline
Process Stage 过程阶段&Percentage of Total Defects 占总缺陷数百分比\\
\hline
RR + DR Requirements review需求评审+ HLD + detailed design review 概要/详细设计评审&15\% (15\% - 20\%) \\
\hline
CR+UT Code reviews代码评审 + unit testing单元测试&55\% (50\% - 70\%) \\
\hline
IT+ST Integration testing集成测试 + system testing系统测试&22\% (20\% - 28\%)  \\
\hline
AT Acceptance testing验收测试&8\% (5\% - 10\%) \\
\hline
\end{tabular}


15



\begin{tabular}{|c|c|}
\hline
SRP 单一职责原则&Single Responsibility Principle \\
\hline
OCP 开放封闭原则&Open Closed Principle \\
\hline
LSP 里氏替换原则&Liskov Substitution Principle \\
\hline
ISP 接口分离原则&Interface Segregation Principle \\
\hline
DIP 依赖倒置原则&Dependency Inversion Principle\\
\hline
\end{tabular}

18


\begin{tabular}{|c|c|c|}
\hline
\:&Conventional传统分工&Composite自主团队\\
\hline
Number of men人数&41&41\\
\hline
Number of segregated tasks 任务数&14&1 \\
\hline
\multirow{3}{*}{Mean job variation for members 每成员的平均任务/变化数: }\\
\hline
Tasks work with 要处理的任务:&1.0&5.5\\
\hline
Main tasks worked 主要任务:&1.0&3.6\\
\hline
Diff. shifts worked 不同的轮班:&2.0&2.9\\
\hline
\end{tabular}



\begin{tabular}{|c|c|c|}
\hline
\:&Conventional传统分工&Composite自主团队\\
\hline
Productivity(\%) 生产率&78&95\\
\hline
Ancillary work at face(hrs per man-shift) 辅助工作(小时/每轮班)&1.32&0.03 \\
\hline
平均后备人力/总人力 (\%)&6&no\\
\hline
Shifts with cycle lag 轮班延迟 (\%)&69&5\\
\hline
最长连续轮班数(没有轮班有问题导致取消)&12&65\\
\hline
\multirow{3}{*}{average per cent of coal won each day平均每天获得的煤炭\%}\\
\hline
\end{tabular}


\begin{tabular}{|c|c|c|}
\hline
\:&Conventional传统分工&Composite自主团队\\
\hline
\multirow{3}{*}{Absenteeism (\% of possible shifts)旷工率(可能轮班数之百分比) }\\
\hline
Without reason 没有理由&4.3&0.4 \\
\hline
Sickness or other 病或其他&8.9&4.6\\
\hline
Accidents 意外&6.8&3.2\\
\hline
Total 总数&20.0&8.2\\
\hline
\end{tabular}

19

\begin{tabular}{|c|c|c|}
\hline
干系人&角色/概况&质量关注重点\\
\hline
商业用户&长期出差者
-坐长途飞机
-做演示
-保护某些秘密文档&-待机时长
-屏幕清晰度
-安全性 \\
\hline
专业媒体&创造性工作;并要协同。
录音和录视频&数字带宽(声音和视频)
电脑速度和内存  \\
\hline
家庭用户&\:&\:\\
\hline
\end{tabular}

22

\begin{tabular}{|c|c|c|c|c|c|c|c|c|c|}
\hline
\multirow{10}{*}{45名学生的测试结果 }\\
\hline
A组(控制组)&17&14&24&20&24&23&16&15&24\\
\hline
B组(控制组)&21&23&13&19&13&19&20&21&16 \\
\hline
C组(赞赏组)&28&30&29&24&27&30&28&28&23\\
\hline
D组(责骂惩罚组)&19&28&26&26&19&24&24&23&22\\
\hline
E组(不理睬组)&21&14&13&19&15&15&10&18&20\\
\hline
\end{tabular}

\begin{tabular}{|c|c|c|c|c|}
\hline
表1&\multirow{4}{*}{头发颜色 }\\
\hline
眼睛颜色&黑&深褐&红&金\\
\hline
棕&68&119&26&7\\
\hline
蓝&20&84&17&94\\
\hline
淡绿褐&15&54&14&10\\
\hline
绿&5&29&14&16\\
\hline
\end{tabular}


\begin{tabular}{|c|c|c|c|c|c|}
\hline
表2&\multirow{5}{*}{头发颜色 }\\
\hline
眼睛颜色&黑&深褐&红&金&总计\\
\hline
棕&68&119&26&7&220\\
\hline
蓝&20&84&17&94&215\\
\hline
淡绿褐&15&54&14&10&93\\
\hline
绿&5&29&14&16&64\\
\hline
总计&108&286&71&127&592\\
\hline
\end{tabular}



\begin{tabular}{|c|c|c|c|c|c|c|c|c|c|}
\hline
表3&\multirow{8}{*}{头发颜色 }&\:\\
\hline
\:&\multirow{2}{*}{黑}&\multirow{2}{*}{深褐}&\multirow{2}{*}{红}&\multirow{2}{*}{金}&\:\\
\hline
眼睛颜色&Obs.&Exp.&Obs.&Exp.&Obs.&Exp.&Obs.&Exp.&Total\\
\hline
棕&68&40&119&106&26&26&7&47&220\\
\hline
蓝&20&39&84&104&17&26&94&46&215\\
\hline
淡绿褐&15&17&54&45&14&11&10&20&93\\
\hline
绿&5&12&29&31&14&8&16&14&64\\
\hline
总计&\multirow{2}{*}{108}&\multirow{2}{*}{286}&\multirow{2}{*}{71}&\multirow{2}{*}{127}&592\\
\hline
\end{tabular}


\begin{tabular}{|c|c|}
\hline
步骤&上面实例 \\
\hline
\:&这些数字代表什么?什么单位(角度、还是距离)? \\
\hline
策划收集数据&怎样取样?避免取样偏差 \\
\hline
确保数据质量&2.5米高女性?异常数据? \\
\hline
初步数据分析(描述性)&平均值、四分位数(Q1、Q2、Q3)、柱状图、箱线图 \\
\hline
数据分析&如果能简单看出来有显著差异,便不需要用假设检验,或其他统计分析方法\\
\hline
解读、沟通结果&当我们读统计分析报告要小心,注意是否被骗 \\
\hline
\end{tabular}


23

\begin{tabular}{|c|c|c|c|c|c|c|c|c|c|}
\hline
\multirow{5}{*}{男生 Males}&\multirow{5}{*}{女生 Females }\\
\hline
6&11&11&8&15&6&8&11&13&8\\
\hline
6&14&8&12&18&7&5&13&14&6\\
\hline
6&9&5&6&9&6&5&5&7&6\\
\hline
6&9&18&7&6&16&10&7&8&5\\
\hline
15&6&11&5&5&16&10&7&8&5\\
\hline
9&9&5&5&8&7&5&5&6&5\\
\hline
9&5&11&5&8&7&8&5&7&6\\
\hline
7&7&5&10&7&11&4&6&8&7\\
\hline
10&7&10&8&11&14&12&5&8&5\\
\hline
\end{tabular}

\begin{tabular}{|c|c|c|c|c|c|c|}
\hline
对象&1&2&3&4&5&6\\
\hline
前(X1)&210&235&208&190&172&244\\
\hline
后(X2)&190&170&210&188&173&228\\
\hline
(X1) - (X2)&20&65&-2&2&-1&16\\
\hline
\end{tabular}


\begin{tabular}{|c|c|c|c|c|}
\hline
布料供货商&1&2&3&4\\
\hline
随机样本拉力&18.5&26.3&20.6&25.4\\
\hline
\:&24.0&25.3&25.2&19.9\\
\hline
\:&17.2&24.0&20.8&22.6\\
\hline
\:&19.9&21.2&24.7&17.5\\
\hline
\:&18.0&24.5&22.9&20.4\\
\hline
样本均值&19.52&24.26&22.84&21.16\\
\hline
样本标准差&2.69&1.92&2.13&2.98\\
\hline
\end{tabular}


\begin{tabular}{|c|c|c|c|c|}
\hline
原因&自由度&离差平方和 (Sum of Squares)&(Mean Square)&F\\
\hline
组间&c - 1&SSA&MSA = SSA/(c-1)&MSA/MSW \\
\hline
组内&n - c&SSW&MSW = SSW /(n-c)&\:\\
\hline
总&n - 1&SST&\:&\:\\
\hline
\end{tabular}

24

\begin{tabular}{|c|c|c|c|c|c|c|c|c|c|c|c|c|c|c|c|c|c|c|c|c|c|c|}
\hline
Component 组件&1&2&3&4&5&6&7&8&9&10&11&12&13&14&15&16&17&18&19&20&21&Totals 总数\\
\hline
Defects 缺陷数&12&16&18&32&22&16&23&35&15&27&16&25&20&26&20&23&23&36&22&27&17&271\\
\hline
Defect Type 类型&\multirow{21}{*}{Number of Defects per Type per Component 缺陷数/类型*组件}&\:\\
\hline
Function 功能&3&5&4&4&4&3&3&20&4&11&2&3&3&5&3&7&4&5&5&15&2&115\\
\hline
Interface 接口&2&2&4&4&3&4&2&3&3&4&2&3&5&3&3&3&2&16&6&2&4&80\\
\hline
Timing 时序&1&1&0&1&1&0&2&1&0&0&2&0&1&1&1&1&1&0&1&0&0&15\\
\hline
Algorithm 算法&0&0&1&14&2&0&0&0&0&0&0&1&5&2&7&6&5&1&2&0&1&47\\
\hline
Checking 检验&1&1&5&1&7&1&1&2&0&1&6&3&1&12&1&0&2&4&3&5&2&59\\
\hline
Assignment 分派&0&2&0&4&1&2&1&3&2&3&2&8&1&0&2&1&2&1&0&1&1&37\\
\hline
Build/Pkg.构建&3&1&1&2&1&0&0&4&3&6&1&0&2&1&1&1&3&2&2&2&1&37\\
\hline
Document 文档&2&4&3&2&3&6&14&2&3&2&1&7&2&2&2&4&4&7&3&2&6&81\\
\hline
\end{tabular}

\end{document}
