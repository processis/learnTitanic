\PassOptionsToPackage{unicode=true}{hyperref} % options for packages loaded elsewhere
\PassOptionsToPackage{hyphens}{url}
%
\documentclass[]{article}
\usepackage{lmodern}
\usepackage{amssymb,amsmath}
\usepackage{ifxetex,ifluatex}
\usepackage{fixltx2e} % provides \textsubscript
\ifnum 0\ifxetex 1\fi\ifluatex 1\fi=0 % if pdftex
  \usepackage[T1]{fontenc}
  \usepackage[utf8]{inputenc}
  \usepackage{textcomp} % provides euro and other symbols
\else % if luatex or xelatex
  \usepackage{unicode-math}
  \defaultfontfeatures{Ligatures=TeX,Scale=MatchLowercase}
\fi
% use upquote if available, for straight quotes in verbatim environments
\IfFileExists{upquote.sty}{\usepackage{upquote}}{}
% use microtype if available
\IfFileExists{microtype.sty}{%
\usepackage[]{microtype}
\UseMicrotypeSet[protrusion]{basicmath} % disable protrusion for tt fonts
}{}
\IfFileExists{parskip.sty}{%
\usepackage{parskip}
}{% else
\setlength{\parindent}{0pt}
\setlength{\parskip}{6pt plus 2pt minus 1pt}
}
\usepackage{hyperref}
\hypersetup{
            pdfborder={0 0 0},
            breaklinks=true}
\urlstyle{same}  % don't use monospace font for urls
\setlength{\emergencystretch}{3em}  % prevent overfull lines
\providecommand{\tightlist}{%
  \setlength{\itemsep}{0pt}\setlength{\parskip}{0pt}}
\setcounter{secnumdepth}{0}
% Redefines (sub)paragraphs to behave more like sections
\ifx\paragraph\undefined\else
\let\oldparagraph\paragraph
\renewcommand{\paragraph}[1]{\oldparagraph{#1}\mbox{}}
\fi
\ifx\subparagraph\undefined\else
\let\oldsubparagraph\subparagraph
\renewcommand{\subparagraph}[1]{\oldsubparagraph{#1}\mbox{}}
\fi

% set default figure placement to htbp
\makeatletter
\def\fps@figure{htbp}
\makeatother


\date{}

\begin{document}

\hypertarget{ux4f8bux5b50ux4e00-ux7b54ux6848ux4e0eux89e3ux8bfb}{%
\subsection{例子一:
答案与解读}\label{ux4f8bux5b50ux4e00-ux7b54ux6848ux4e0eux89e3ux8bfb}}

共三个实体:

\begin{enumerate}
\tightlist
\item
  合同员工
\item
  管理设施
\item
  轮班
\end{enumerate}

你可能会问:那些合同员工的职称是否也应该是一个实体?(因需要花工夫开发)\\
这不应该是一个实体,原因:人员的职称必须依赖人员的信息挂在一起,不可以独立存在,就好比我们要维护员工信息,假如也要维护员工的家属信息,这个家属信息就不能算另外一个实体,因为没有人员的话,家属是不能单独存在的。原则:不是根据是否要产生开发的工作量,而是从用户角度看,这个实体能否独立存在和维护。否则功能点的估算就只是根据个人对开发工作量的估计,而不是从用户角度看功能的客观判断。

每个实体对用户来讲,都有新增、展示、修改、删除4个功能。在人员管理里,还有一个功能是显示一个可选的列表,方便用户选择,这功能是增查改删以外的第五个功能。

设施管理也同样有这个列可选设备设施的一个展示框这第五个功能。

在轮班管理里面,除了增加、查看、修改、删除和展示外,它里面有两个下拉框功能:

\begin{enumerate}
\tightlist
\item
  让客户挑选相关设施的 Combo-box下拉框
\item
  让客户选人员的框
\end{enumerate}

你可能会问,这 2
个下拉框功能是否不应该算额外的功能,而是属于``轮班''的增删改查基本功能的一部分?\\
我们可以这样想:从用户的角度来看,如果没有这两个下拉框的功能,基本的增删改查功能是否可以实现;现在做了两个下拉框的功能,是额外的新增功能,更方便用户去选择,所以这两个算是额外两个功能。

也可参考IFPUG关于EI/EO/EQ 的识别要求;基本操作(elementary
process)必须符合以下三条之一:

\begin{enumerate}
\tightlist
\item
  使用独特处理逻辑,与应用中其他`行为'(EI/EO/EQ) 的处理逻辑不同
\item
  在该处理中识别出来的数据元素是与应用中其他`行为'(EI/EO/EQ)
  的数据元素不同
\item
  在该处理中引用的`实体'(ILF 和EIF) 与其他`行为'(EI/EO/EQ)
  所引用的不同
\end{enumerate}

它要列出所有合条件的数据元素进这下拉框,类似一个新的报表,所以算一个行为。基于同类原因,挑选相关设施的
Combo-box 下拉框,选择可用合同员工 Combo-box 下拉框, 等各自也算一个行为。

在轮班里,还有一个展示可选的轮班功能,另外是分配轮班功能。还有最后的
RF05 六个查询功能。

得出共 24(=5+5+6+2+6)行为,加 3实体, 所以按简化功能点每个实体 x7,每行为
x4.6 得出,共新增131.4(=3x7+24x4.6)简化功能点,详见下面列表:

\href{文件:Ex1SoluScreenshot_2022-04-05_115926.jpg}{500px}

\href{文件:微信截图_20220412130822.jpg}{550px}

\hypertarget{ux8ba1ux7b97ux529fux80fdux89c4ux6a21}{%
\subsubsection{计算功能规模}\label{ux8ba1ux7b97ux529fux80fdux89c4ux6a21}}

\begin{description}
\item[]
\begin{description}
\tightlist
\item[]
DSFP = ADD + CFP
\end{description}
\end{description}

因为没有数据转换,所以 CFP=0, 所以 DSFP = (110.4+21) +0 = 131.4 SiFP

因是首次开发, ASFP = ADD = 131.4 SiFP

\hypertarget{ux6f5cux6c34ux5b66ux6821femux9879ux76ee}{%
\subsection{2.潜水学校:FEM项目}\label{ux6f5cux6c34ux5b66ux6821femux9879ux76ee}}

\hypertarget{ux63cfux8ff0}{%
\subsubsection{描述}\label{ux63cfux8ff0}}

参照之前的潜水学校系统,对功能进行了增强,并提出了该软件的功能优化维护项目(FEM)。

\hypertarget{ux529fux80fdux9700ux6c42}{%
\subsubsection{功能需求}\label{ux529fux80fdux9700ux6c42}}

\hypertarget{rf01}{%
\paragraph{RF01}\label{rf01}}

用户想要取消合同员工删除功能。The user wants to eliminate the Contractor
delete function.

\hypertarget{rf02}{%
\paragraph{RF02}\label{rf02}}

在短途潜水里, 在潜水设施管理中能管理船上医生的存在或缺失。The
presence/absence of a ship doctor during excursions must be managed in
the file DIVING FACILITIES.

\hypertarget{rf03}{%
\paragraph{RF03}\label{rf03}}

出于税收和安全原因,不再需要删除可用轮班这项功能 For tax and safety
reasons the function to delete availability shifts will no longer be
required.

\hypertarget{rf04}{%
\paragraph{RF04}\label{rf04}}

用户还需要管理课程参与者信息和他们参加的那个短途潜水信息:\\
*管理参与者的信息包括:参与者ID,姓,名,出生日期,潜水执照,执照日期。\\
参加短途潜水:参与者ID。轮班编号,出游日,天数,最终考试是/否通过\\
*用列表框 (包括:参与者ID,姓,名)来选择轮班中的参与者。\\
*使用原本应用程序中已经有的列表框选择轮班。\\
*用功能键将激活这功能,并最终生成错误信息/结果。

用功能键初始填充课程参与者信息,参与者信息源自以前参与者信息的备份数据。

\hypertarget{rf05}{%
\paragraph{RF05}\label{rf05}}

用户还需要能够在课程结束时颁发出席证书给在短途潜水中登记的所有参与者。除了管理参与者基础数据外,还需要管理:参与者所登记的轮班、轮班日期、时长、教练的姓名和医生(如在场)的姓名。该功能使用原本应用程序中已经可用的功能:选择轮班。用功能键将激活这些功能,并最终生成错误/结果消息。

\hypertarget{rf06}{%
\paragraph{RF06}\label{rf06}}

用户还需要能够向合同员工颁发``教员身份参与证书'',其中的信息除了基础数据外还包括:轮班ID、教练ID、出游日、船医(如果有的话)。对于轮班选择,将使用原本应用程序中已经有的列表框。用功能键将激活这功能,并最终生成错误信息/结果。

\hypertarget{ux4f8bux5b50ux4e8c-ux7b54ux6848ux4e0eux89e3ux8bfb}{%
\subsection{例子二:
答案与解读}\label{ux4f8bux5b50ux4e8c-ux7b54ux6848ux4e0eux89e3ux8bfb}}

RF02 变动了潜水设施的内容,所以设施实体有变更。\\
因为设施的信息有变更,导致跟这实体相关的行为,包括新增、编辑、和展示这三行为都会有变更。

另外加了两个要管理的实体:

\begin{enumerate}
\tightlist
\item
  参与者
\item
  短途潜水
\end{enumerate}

不需要合同员工的删除功能,所以是个行为删除。

在参与者的管理,除了增加,改动,展示和删除四个功能以外,还有可以挑选参与者的下拉框功能。

两个证书的功能

\begin{enumerate}
\tightlist
\item
  给教练的证书
\item
  给参与者发证书
\end{enumerate}

对应每个短途潜水也需要有添加、改动、展示、删除的四功能。
那个删除轮班功能也被删掉了。

增加了两个实体 -\/-参与者 与 短途潜水旅行登记\\
Q: 为什么短途潜水旅行登记算一个实体?\\
A: 因它包括的信息都不能归入已有的 【参与者】 【合同员工】 【设施
】【轮班】实体里,例如那位参与者参加了那个班,考试分数等。
也可参考IFPUG关于ILF/EIF (实体)的识别要求;必须符合以下条件:

\begin{enumerate}
\tightlist
\item
  数据的集合必须是逻辑相关的并且是用户可以识别
\item
  这些数据或者控制信息必须是在本应用的边界内被维护
\end{enumerate}

总结:

\begin{itemize}
\tightlist
\item
  实体方面增加了2 实体; 设施实体有变更。
\item
  行为方面主要的在短途潜水旅行方面增加了4 增删改查的功能和。5
  参与者的功能(因为在里面加了一个下拉框功能),增加了2
  证书功能。改动了设施的增加、编辑、和展示,三个行为,删掉了两个行为。
\end{itemize}

所以动态功能点是增加的功能点64.6 (=2x7 +(4+2+5)x4.6),变更 20.8
(=3x4.6),删除9.2 (=2x4.6),总共的动态简化功能点 94.6。

静态功能点依据上面练习一那的131.4,加上增加的功能点 64.6,减掉
删除功能点 9.2,得出变更后静态功能点 186.8。

\href{文件:Ex2XlsScreenshot_2022-04-05_143941.jpg}{550px}

\href{文件:Ex2XlsPt2of2Screenshot_2022-04-05_143941.jpg}{550px}

\hypertarget{ux8ba1ux7b97ux529fux80fdux89c4ux6a21-1}{%
\subsubsection{计算功能规模}\label{ux8ba1ux7b97ux529fux80fdux89c4ux6a21-1}}

\begin{description}
\item[]
\begin{description}
\tightlist
\item[]
ESFP = ADD + CHG + DEL + CFP
\end{description}
\end{description}

因为有数据转换:初始填充课程参与者信息作为一个基本过程,所以 CFP=4.6

\begin{description}
\tightlist
\item[]
ESFP = (64.6 + 20.8 + 9.2) + 4.6 = 94.6 + 4.6 = 99.2 SiFP
\end{description}

软件开发后的静态功能点: ASFPA = ASFPB + ADD - DEL = 131.4 + 64.6 - 9.2
= 186.8 SiFP

\end{document}
