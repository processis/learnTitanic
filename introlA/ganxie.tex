\chapter*{感谢 Acknowledgments } % Introduction chapter suppressed from the table of contents

6年前开始把一些培训、评估经验,配合软件工程/项目管理知识,写分享文章,在公司网站和CSDN上发布;2018年教ACP敏捷课,加深了我对敏捷的了解,也发现很多人未了解敏捷开放的重点。初始时候缺乏经验,虽然有些很有趣的题材,但未能组成系列性文章,后面逐步某注销组成系列,分享文章也越来越多被转载。2022年受疫情令影响,少出差,可以有更多时间把文章组成书,幸运有各朋友的帮助,终于可以在2023年出版。

非常感谢我的老师、客户、学生和朋友们,如果没有与他们面对面详细讨论,并在项目过程中反复试验与反馈,不可能写出这本书。这些宝贵经验帮我验证了各种敏捷思路与量化管理的可用性,也让我更加清晰地理解了质量改进的重点。

也感谢朋友圈里各位老师、行业精英们分享经验和意见,其中包括北京的杨立东、王绍军、纪钟涛、武宏伟,天津的韩淑军,上海的周青龙,杭州的程文进、 胡蕊莉,成都的杨杰等; 感谢杭州的徐洪洋,北京的洪一海、赵丽在百忙之中抽空提出了文字上的建议;感谢无锡的陈镜庆、秦瑜不断提出修改意见,并帮助我最后完成本书的编辑;也感谢我香港大学老同学李启良先生对Mediawiki服务器、Linux、LATEX 等平台的技术支持。 