\chapter*{前言 Prologue} % Introduction chapter suppressed from the table of contents

\begin{quote}
We are uncovering better ways of developing software by doing it and helping others do it.\\
--Agile Software Development Manifesto
\end{quote}

敏捷宣言已经有超过二十年的历史,国内越来越多软件开发团队开始采用敏捷和迭代,但总体效果参差不齐。有些年轻的团队成员听到敏捷不需要文档,以为也不需要注重代码质量,包括代码可读性,导致后面发布的软件产品问题多多,难以维护。要编写出高质量的代码,人本身能力非常关键,但软件工程快速发展,导致编程员人数快速增长,其中很多缺乏专业工程师素养,所有若要敏捷开发真正起作用必须先提升编程员能力。\\
没有数据就无法管理,但很多敏捷团队只是走流程(每天站立会议、看板并非敏捷的重点),缺乏度量,所以管理者一听到敏捷就觉得不靠谱,要求团队用回传统的瀑布开发方式。\\
这二十年来,中外都出版了很多关于敏捷的书,但绝大部分都没有深入去探索以上的问题。这本书就是希望通过解读各种敏捷最佳实践,加上敏捷以外的其他知识,帮助大家理解并更好使用敏捷,提升软件的质量和总生产率,让团队成员与公司管理层获得双赢。\\
同时也希望管理层通过这本书能了解敏捷开发的要素,并能使用敏捷开发模式,帮助公司提升软件产品质量,同时降低成本增加公司的竞争力。\\

\framebox{%
\begin{minipage}[t]{0.97\columnwidth}\raggedright
某技术总监陈总在5天差距分析的最终总结会里问开发团队:

\begin{enumerate}
\tightlist
\item
  你们知道自己是每天产出多少代码行吗?(生产率)
\item
  你们知道平均修复一个系统测试的缺陷花多少工作量(人时)?
\item
  有没有发生过:代码开发完成,系统测试人员尝试运行,但跑不动?(开发人员不能借口说不懂业务,或需求不清,因为你写完代码后,自己连最基本确保代码能运行都没有去做。我还未问你们做开发的有没有做好单元测试/集成测试。)
\item
  你们做测试的知道测试覆盖率是多少?
\end{enumerate}

陈总有丰富开发经验,他最后总结说:
``我以前自己做开发时也是被问过这些问题,我当时也不懂,但我能知耻而后勇------先知道现在的不足,然后按评估老师的最佳实践,持续改进。''\strut
\end{minipage}}

\begin{itemize}
\tightlist
\item
  如果你像陈总下面团队一样,不懂以上问题的答案,这手册可以帮你自己找答案。
\item
  如果你(是管理者或客户/用户)不满意团队的软件开发成果,这手册可以帮你找出根因,启发团队改善。
\item
  如果你已经很了解,恭喜你,但应还可能从这手册里找到一些你未知但有用的方法。
\end{itemize}

\framebox{%
\begin{minipage}[t]{0.97\columnwidth}\raggedright
1980年,STROUSTRUP
先生不满意当时很流行的C语言,因为在70年代,已经有如SmallTalk之类的面向对象语言,C是基于传统步骤式的语言,没有面向对象的功能,所以STROUSTRUP先生就基于C的基础,加上有类功能的C
(C with
Classes)。过了4年,他同事觉得这个名字不好,就改成C++,也出了一本书叫C++,与本来经典的C很相似。后面C++就成为了面向对象的常用语言。\\ A++
继承40年前
C++的思路,希望在敏捷(Agile)开发的基础上,加上增值的元素,更能帮助团队与管理者做好软件开发,让客户满意。\strut
\end{minipage}}
