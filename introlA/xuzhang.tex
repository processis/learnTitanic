\chapter*{序} % Introduction chapter suppressed from the table of contents


杨立东 《微管理》作者,四维世景科技(北京)有限公司 总经理

这是一本从事敏捷开发和过程改进的人员的必修之书,作为本书的早期读者,有两个创新让我印象深刻。其一是本书的写法,类似《金刚经》体,全文多是以对话的方式,让读者读起来轻松流畅,少了长篇大论的说教,而是真实和不同IT企业高管、中层、甚至开发团队的问答,在问答中将普遍问题进行归纳总结,并提出解决的方法。其二是案例和数据的引用,虽然很多作者都在书籍中应用案例和大量数据,宋老师则是结合自己多年的从业经验,对案例和数据精益求精,每个案例都颇具经典。通过阅读和理解这些案例,更能体会对话中的问题,以及形成的解决方案的建议。

读过很多管理类的书籍,有些管理数据自始至终贯穿一条主线,例如德鲁克的《卓有成效的管理者》,有些书籍则以故事的形式给企业管理做出启示,例如《目标》。本书的内容始终贯穿过程改进的主题,将宋老师多年在该领域的咨询和培训经验跃然纸上,其中也引用了很多小故事,让书籍活跃了起来,让读者读起来不至于太累。诚然,一千个人有一千个哈姆雷特,具体大家在书籍阅读中去体会吧。

给我感触更深的是宋老师的严谨的写作态度,每次他写完一章,都发给不同的管理者去阅读和提出改进的建议,每次的问题都会得到反馈和在章节中得以应用。这种态度是非常值得尊敬的,毕竟在该领域他才是货真价实的专业人士,而我们这些先期读者只是普通的从业者。

最后,预祝书籍能帮助到那些有志于在管理上提升的从业人员。

\begin{description}
\item[]
\begin{description}
\tightlist
\item[]
2023年3月于北京
\end{description}
\end{description}


\chapter*{感谢 Acknowledgments } % Introduction chapter suppressed from the table of contents

6年前开始把一些培训、评估经验,配合软件工程/项目管理知识,写分享文章,在公司网站和CSDN上发布;2018年在香港教ACP敏捷课也加深了我对敏捷的了解,也发现很多人未了解敏捷开放的重点。初始时候缺乏经验,虽然有些很有趣的题材,但未能组成系列性文章,后面逐步某注销组成系列,分享文章也越来越多被转载。2022年受疫情影响,少出差,可以有更多时间把文章组成书,幸运有各朋友的帮助,终于可以在2023年出版。

非常感谢我的老师、客户、学生和朋友们,如果没有与他们面对面的详细讨论,并在项目过程中反复试验与反馈,不可能写出这本书。这些宝贵经验帮我验证了各种敏捷思路与量化管理的可用性,也让我更加清晰地理解了质量改进的重点。

也感谢朋友圈里各位老师、行业精英们分享经验和意见:其中包括北京的杨立东、王绍军、纪钟涛、武宏伟,天津的韩淑军,上海的周青龙,杭州的胡蕊莉,成都的杨杰等。
感谢杭州的徐洪洋在百忙之中抽空提出了文字上的建议;感谢无锡的陈镜庆、秦瑜不断提出修改意见,并帮助我最后完成本书的编辑;也感谢香港李启良先生对
Mediawiki服务器,Linux,LATEX 等平台的技术支持。