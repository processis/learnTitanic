\chapter*{序 Preface} % Introduction chapter suppressed from the table of contents


这是一本从事敏捷开发和过程改进的人员的必修之书,作为本书的早期读者,有两个创新让我印象深刻。其一是本书的写法,类似《金刚经》体,全文多是以对话的方式,让读者读起来轻松流畅,少了长篇大论的说教,而是加入了真实的、与不同IT企业高管、中层及基层开发和技术人员的对话,在问答中将普遍问题进行归纳总结,并提出解决方法。其二是案例和数据的引用,虽然很多是引用书籍中的案例和数据,宋老师能结合自己多年经验,对案例和数据的引用力求精益求精,每个案例都颇为经典。阅读和理解这些案例能加深对会话中问题的领悟,也加深了对相关建议和解决方案的理解。

读过很多管理类的书籍,有些书籍将管理数据自始至终贯穿如一,例如德鲁克的《卓有成效的管理者》;有的则以故事的形式给企业管理者以启示,例如《目标》。而本书的内容始终如一地贯穿了过程改进的主题,通过引用一些小故事,宋老师将多年来在该领域的培训和咨询的经验跃然纸上,让书籍活跃起来,使读者读起来不至于太累。诚然,一千个人有一千个哈姆雷特,具体有什么收获,就请大家在本书的阅读中去体会吧。

除此以外,给我感触更深的是宋老师严谨的写作态度,每次他写完一章,都会发给不同的管理者阅读,邀请他们提供改进的意见和建议,每一个建议都会得到反馈并在相关章节中加以完善。这种态度是非常值得尊敬的,毕竟在他才是该领域货真价实的专业人士,而我们这些先期读者只是普通的从业者。
最后,预祝本书能帮助到那些有志于在管理上提升的从业人员。 

\begin{description}
\item[]
\begin{description}
\tightlist
\item[]
杨立东 《微管理》作者,四维世景科技(北京)有限公司总经理

\begin{description}
\item[]
\begin{description}
\tightlist
\item[]
2023年3月于北京
\end{description}
\end{description}
\end{description}
\end{description}

